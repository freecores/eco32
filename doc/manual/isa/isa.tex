\newenvironment{effectize}{Effect: \begin{itemize}}{\end{itemize}}
\newcommand{\effect}{\item[]}
\newenvironment{effectblock}{\begin{itemize}}{\end{itemize}}

\newcommand{\regeffect}[1]{\effect $R_r \leftarrow #1$}
\newcommand{\regeffects}[1]{\begin{effectize}\regeffect{#1}\end{effectize}}

\newcommand{\bitregeffect}[1]{\effect $R_{r,i} \leftarrow #1$}
\newcommand{\bitregeffects}[1]{\begin{effectize}\bitregeffect{#1}\end{effectize}}

\newcommand{\priveffect}{\effect if $U_C = 1$ then trigger a \name{Privileged Instruction Fault}}

\newcommand{\rrrformat}[1]{
Format:
\begin{tabular}{|c|c|c|c|c|c|}
\hline
Bits & 31..26 & 25..21 & 20..16 & 15..11 & 10..0\\
\hline
Value & #1 & x & y & r & (ignored)\\
\hline
\end{tabular}
}

\newcommand{\rriformat}[1]{
Format:
\begin{tabular}{|c|c|c|c|c|}
\hline
Bits & 31..26 & 25..21 & 20..16 & 15..0\\
\hline
Value & #1 & x & r & y\\
\hline
\end{tabular}
}

\newcommand{\jformat}[1]{
Format:
\begin{tabular}{|c|c|c|}
\hline
Bits & 31..26 & 25..0\\
\hline
Value & #1 & offset\\
\hline
\end{tabular}
}

\newcommand{\jrformat}[1]{
Format:
\begin{tabular}{|c|c|c|c|}
\hline
Bits & 31..26 & 25..21 & 20..0\\
\hline
Value & #1 & dest & (ignored)\\
\hline
\end{tabular}
}

\newcommand{\brformat}[1]{
Format:
\begin{tabular}{|c|c|c|c|c|}
\hline
Bits & 31..26 & 25..21 & 20..16 & 15..0\\
\hline
Value & #1 & x & y & offset\\
\hline
\end{tabular}
}

\newcommand{\ldstformat}[1]{
Format:
\begin{tabular}{|c|c|c|c|c|}
\hline
Bits & 31..26 & 25..21 & 20..16 & 15..0\\
\hline
Value & #1 & x & r & y\\
\hline
\end{tabular}
}

\newcommand{\noargformat}[1]{
Format:
\begin{tabular}{|c|c|c|}
\hline
Bits & 31..26 & 25..0\\
\hline
Value & #1 & (ignored)\\
\hline
\end{tabular}
}

\newcommand{\mvspformat}[1]{
Format:
\begin{tabular}{|c|c|c|c|c|}
\hline
Bits & 31..26 & 25..21 & 20..16 & 15..0\\
\hline
Value & #1 & (ignored) & r & z\\
\hline
\end{tabular}
}

\chapter{Instruction Set}

The instructions of the \eco operate directly on the functional components described in the previous chapter. They can be subdivided into groups of instructions that work in a similar way:
\begin{itemize}
\item {\it Computation}: These instructions compute a function of values stored in general-purpose registers or encoded directly into the instruction and store the result in a general-purpose register.
\item {\it Control Flow}: These instructions affect the \pc in various ways.
\item {\it Load/Store}: These instructions transfer data from or to RAM locations or peripheral device registers.
\item {\it System}: Special instructions for \pswx, \mmux, or exception operation.
\end{itemize}

\section{Definitions}

Some definitions are useful when explaining the effect of an instruction: An \definition{immediate} value is a value encoded directly into the instruction. A \definition{register value} is a 32-bit value taken from a general-purpose register. The interpretation of such values is up to the instruction.

A register value is referred to by an instruction by an immediate value that denotes the register number. If $x$ is a 5-bit immediate value, then $R_x$ shall denote the corresponding register value, and $R_x \leftarrow ...$ shall denote an assignment to this register. Similarly, $S_i$ denotes special purpose register \#i. $R_{i,j}$ and $S_{i,j}$ denote specific bits of a register. As a special rule, an assignment to $R_0$ has no effect since that register is not writeable.

\section{General Execution Loop}

The \eco executes the following loop to perform its task:
\begin{itemize}
\item Remember the current value of the \pc register. If any exception occurs before the instruction is finished, this value is placed in register \#30 such that the current instruction can be restarted.
\item Load the current instruction from the virtual address stored in the \pcx. If that address is not word-aligned, then an \name{Invalid Address Exception} occurs. Otherwise, if it is a privileged address and the CPU is in user mode, then a \name{Privileged Address Exception} occurs. Otherwise, it is mapped to a physical address by the \mmux, which may trigger a \name{TLB Miss Exception} or a \name{TLB Invalid Exception}. All these exceptions cause the faulting \pc value to be stored in the \name{TLB Bad Address Register}. Note that a \name{TLB Write Exception} cannot occur since the instruction fetch is a read access.
\item Increase the \pc by 4.
\item If the opcode in the instruction word does not denote a valid instruction, then an \name{Illegal Instruction Fault} is triggered.
\item Decode and execute the instruction. Any fault triggered during this step immediately stops execution of the current instruction and transfers control to the fault service routine.
\item Remember the new value of the \pc register. If any interrupt occurs in the next step, this value is placed in register \#30 such that control can return to the next instruction.
\item Test for interrupts. If an interrupt is signalled and admitted (\myref{2}{ien}), then control is transferred to the service routine (\myref{2}{accept_exception}).
\end{itemize}

\section{Computation Instructions}

\newcommand{\rdivzero}{\effect if $R_y=0$ then trigger a \name{Division by Zero Fault}}
\newcommand{\idivzero}{\effect if $y=0$ then trigger a \name{Division by Zero Fault}}

The computation instructions compute a function of register values and/or immediate values, and store their result in a general-purpose register.

\subsection{ADD}

The ADD instruction computes the sum of two 32-bit register operands, truncated to 32 bits.\\

\rrrformat{000000}

\regeffects{truncate_{32}(R_x + R_y)}

\subsection{ADDI}

The ADDI instruction computes the sum of a 32-bit register operand and a sign-extended 16-bit immediate operand, truncated to 32 bits.\\

\rriformat{000001}

\regeffects{truncate_{32}(R_x + signext_{32}(y))}

\subsection{SUB}

The SUB instruction computes the difference of two 32-bit register operands, truncated to 32 bits.\\

\rrrformat{000010}

\regeffects{truncate_{32}(R_x - R_y)}

\subsection{SUBI}

The SUBI instruction computes the difference of a 32-bit register operand and a sign-extended 16-bit immediate operand, truncated to 32 bits.\\

\rriformat{000011}

\regeffects{truncate_{32}(R_x - signext_{32}(y))}

\subsection{MUL}

The MUL instruction computes the signed product of two 32-bit register operands, truncated to 32 bits.\\

\rrrformat{000100}

\regeffects{truncate_{32}(R_x *_{signed} R_y)}

\subsection{MULI}

The MULI instruction computes the signed product of a 32-bit register operand and a sign-extended 16-bit immediate operand, truncated to 32 bits.\\

\rriformat{000101}

\regeffects{truncate_{32}(R_x *_{signed} signext_{32}(y))}

\subsection{MULU}

The MULU instruction computes the unsigned product of two 32-bit register operands, truncated to 32 bits.\\

\rrrformat{000110}

\regeffects{truncate_{32}(R_x *_{unsigned} R_y)}

\subsection{MULUI}

The MULUI instruction computes the unsigned product of a 32-bit register operand and a zero-extended 16-bit immediate operand, truncated to 32 bits.\\

\rriformat{000111}

\regeffects{truncate_{32}(R_x *_{unsigned} zeroext_{32}(y))}

\subsection{DIV}

The DIV instruction computes the signed quotient of two 32-bit register operands, truncated to 32 bits.\\

\rrrformat{001000}

\begin{effectize}
\rdivzero
\regeffect{truncate_{32}(R_x /_{signed} R_y)}
\end{effectize}

\subsection{DIVI}

The DIVI instruction computes the signed quotient of a 32-bit register operand and a sign-extended 16-bit immediate operand, truncated to 32 bits.\\

\rrrformat{001001}

\begin{effectize}
\idivzero
\regeffect{truncate_{32}(R_x /_{signed} signext_{32}(y))}
\end{effectize}

\subsection{DIVU}

The DIVU instruction computes the unsigned quotient of two unsigned 32-bit register operands, truncated to 32 bits.\\

\rrrformat{001010}

\begin{effectize}
\rdivzero
\regeffect{truncate_{32}(R_x /_{unsigned} R_y)}
\end{effectize}

\subsection{DIVUI}

The DIVUI instruction computes the unsigned quotient of a 32-bit register operand and a zero-extended 16-bit immediate operand, truncated to 32 bits.\\

\rrrformat{001011}

\begin{effectize}
\idivzero
\regeffect{truncate_{32}(R_x /_{unsigned} zeroext_{32}(y))}
\end{effectize}

\subsection{REM}

The REM instruction computes the signed remainder of two 32-bit register operands, truncated to 32 bits.\\

\rrrformat{001100}

\begin{effectize}
\rdivzero
\regeffect{truncate_{32}(R_x MOD_{signed} R_y)}
\end{effectize}

\subsection{REMI}

The REMI instruction computes the signed remainder of a 32-bit register operand and a sign-extended 16-bit immediate operand, truncated to 32 bits.\\

\rrrformat{001101}

\begin{effectize}
\idivzero
\regeffect{truncate_{32}(R_x MOD_{signed} signext_{32}(y))}
\end{effectize}

\subsection{REMU}

The REMU instruction computes the unsigned remainder of two unsigned 32-bit register operands, truncated to 32 bits.\\

\rrrformat{001110}

\begin{effectize}
\rdivzero
\regeffect{truncate_{32}(R_x MOD_{unsigned} R_y)}
\end{effectize}

\subsection{REMUI}

The REMUI instruction computes the unsigned remainder of a 32-bit register operand and a zero-extended 16-bit immediate operand, truncated to 32 bits.\\

\rrrformat{001111}

\begin{effectize}
\idivzero
\regeffect{truncate_{32}(R_x MOD_{unsigned} zeroext_{32}(y))}
\end{effectize}

\subsection{AND}

The AND instruction computes the bitwise AND of two 32-bit register operands.\\

\rrrformat{010000}

\bitregeffects{R_{x,i} \wedge R_{y,i}}

\subsection{ANDI}

The ANDI instruction computes the bitwise AND of a 32-bit register operand and a zero-extended 16-bit immediate operand.\\

\rriformat{010001}

\bitregeffects{R_{x,i} \wedge zeroext_{32}(y)_i}

\subsection{OR}

The OR instruction computes the bitwise OR of two 32-bit register operands.\\

\rrrformat{010010}

\bitregeffects{R_{x,i} \vee R_{y,i}}

\subsection{ORI}

The ORI instruction computes the bitwise OR of a 32-bit register operand and a zero-extended 16-bit immediate operand.\\

\rriformat{010011}

\bitregeffects{R_{x,i} \vee zeroext_{32}(y)_i}

\subsection{XOR}

The XOR instruction computes the bitwise XOR of two 32-bit register operands.\\

\rrrformat{010100}

\bitregeffects{R_{x,i} \oplus R_{y,i}}

\subsection{XORI}

The XORI instruction computes the bitwise XOR of a 32-bit register operand and a zero-extended 16-bit immediate operand.\\

\rriformat{010101}

\bitregeffects{R_{x,i} \oplus zeroext_{32}(y)_i}

\subsection{XNOR}

The XNOR instruction computes the bitwise XNOR of two 32-bit register operands.\\

\rrrformat{010110}

\bitregeffects{\overline{R_{x,i} \oplus R_{y,i}}}

\subsection{XNORI}

The XNORI instruction computes the bitwise XNOR of a 32-bit register operand and a zero-extended 16-bit immediate operand.\\

\rriformat{010111}

\bitregeffects{\overline{R_{x,i} \oplus zeroext_{32}(y)_i}}

\subsection{SLL}

The SLL instruction computes the result of shifting the first 32-bit register operand to the left by a number of bits specified by the 5 least significant bits of the second 32-bit register operand, and filling up with 0 bits.\\

\rrrformat{011000}

\begin{effectize}
\effect $shift \leftarrow unsigned(R_{y,4..0})$
\effect $temp_i \leftarrow R_{x,i-shift}$ if $i \geq shift$
\effect $temp_i \leftarrow 0$ if $i < shift$
\effect $R_r \leftarrow temp$
\end{effectize}

\subsection{SLLI}

The SLLI instruction computes the result of shifting the 32-bit register operand to the left by a number of bits specified by the 5 least significant bits of the immediate operand, and filling up with 0 bits.\\

\rriformat{011001}

\begin{effectize}
\effect $shift \leftarrow unsigned(y_{4..0})$
\effect $temp_i \leftarrow R_{x,i-shift}$ if $i \geq shift$
\effect $temp_i \leftarrow 0$ if $i < shift$
\effect $R_r \leftarrow temp$
\end{effectize}

\subsection{SLR}

The SLR instruction computes the result of shifting the first 32-bit register operand to the right by a number of bits specified by the 5 least significant bits of the second 32-bit register operand, and filling up with 0 bits.\\

\rrrformat{011010}

\begin{effectize}
\effect $shift \leftarrow unsigned(R_{y,4..0})$
\effect $temp_i \leftarrow R_{x,i+shift}$ if $i + shift < 32$
\effect $temp_i \leftarrow 0$ if $i + shift \geq 32$
\effect $R_r \leftarrow temp$
\end{effectize}

\subsection{SLRI}

The SLRI instruction computes the result of shifting the 32-bit register operand to the right by a number of bits specified by the 5 least significant bits of the immediate operand, and filling up with 0 bits.\\

\rriformat{011011}

\begin{effectize}
\effect $shift \leftarrow unsigned(y_{4..0})$
\effect $temp_i \leftarrow R_{x,i+shift}$ if $i + shift < 32$
\effect $temp_i \leftarrow 0$ if $i + shift \geq 32$
\effect $R_r \leftarrow temp$
\end{effectize}

\subsection{SAR}

The SAR instruction computes the result of shifting the first 32-bit register operand to the right by a number of bits specified by the 5 least significant bits of the second 32-bit register operand, and replicating the topmost (sign) bit.\\

\rrrformat{011100}

\begin{effectize}
\effect $shift \leftarrow unsigned(R_{y,4..0})$
\effect $temp_i \leftarrow R_{x,i+shift}$ if $i + shift < 32$
\effect $temp_i \leftarrow R_{x,31}$ if $i + shift \geq 32$
\effect $R_r \leftarrow temp$
\end{effectize}

\subsection{SARI}

The SARI instruction computes the result of shifting the 32-bit register operand to the right by a number of bits specified by the 5 least significant bits of the immediate operand, and replicating the topmost (sign) bit.\\

\rriformat{011101}

\begin{effectize}
\effect $shift \leftarrow unsigned(y_{4..0})$
\effect $temp_i \leftarrow R_{x,i+shift}$ if $i + shift < 32$
\effect $temp_i \leftarrow R_{x,31}$ if $i + shift \geq 32$
\effect $R_r \leftarrow temp$
\end{effectize}

\subsection{LDHI}

The LDHI instruction is used to generate large constants. The upper 16 bits of the result are taken from the 16-bit immediate operand. The lower 16 bits of the result are 0.\\

\rriformat{011111}

\begin{effectize}
\effect $R_{r,31..16} \leftarrow y_{15..0}$
\effect $R_{r,15..0} \leftarrow 0$
\end{effectize}

\section{Control Flow Instructions}

Control flow instruction load immediate values or register values into the \pc and/or load the value of the \pc into a general-purpose register. The \eco supports unconditional {\it jumps}, conditional {\it branches}, indirect jumps, subroutine calls, subroutine returns, and indirect subroutine calls out of the box. More complex control flow schemes can be implemented by combining these instructions.

A control transfer is \definition{conditional} if it only occurs on a certain condition that is computed from general-purpose registers. A control transfer is \definition{unconditional} if it always occurs.

A control transfer is \definition{direct} if the target address is supplied as an immediate value. It is \definition{indirect} if the target address is supplied as a register value.

A control transfer is \definition{absolute} if the value of the \pc is overwritten with a totally new value. It is \definition{relative} if the value of the \pc is modified by adding or subtracting an offset.

Both relative control transfers and instructions that read the current \pc value operate on the value of the \pc {\it after} increasing it by 4 during instruction fetching.

\newcommand{\branchdesc}[4]{
\subsection{#1}

The #1 instruction performs a conditional direct jump to a relative immediate sign-extended 16-bit offset counted as words. The condition is evaluated by comparing two 32-bit register operands and is asserted if the first operand is #3 the second operand.

\brformat{#2}

\begin{effectize}
\effect if $R_x #4 R_y$ then $PC \leftarrow PC + 4 * signext_{32}(offset)$
\end{effectize}

}

\branchdesc{BEQ}{100000}{equal to}{=}
\branchdesc{BNE}{100001}{not equal to}{\neq}
\branchdesc{BLE}{100010}{less or equal to (by signed comparison)}{\leq_{signed}}
\branchdesc{BLEU}{100011}{less or equal to (by unsigned comparison)}{\leq_{unsigned}}
\branchdesc{BLT}{100100}{less than (by signed comparison)}{<_{signed}}
\branchdesc{BLTU}{100101}{less than (by unsigned comparison)}{<_{unsigned}}
\branchdesc{BGE}{100110}{greater or equal to (by signed comparison)}{\geq_{signed}}
\branchdesc{BGEU}{100111}{greater or equal to (by unsigned comparison)}{\geq_{unsigned}}
\branchdesc{BGT}{101000}{greater than (by signed comparison)}{>_{signed}}
\branchdesc{BGTU}{101001}{greater than (by unsigned comparison)}{>_{unsigned}}
\subsection{J}

The J instruction performs an unconditional direct jump to a relative immediate sign-extended 26-bit offset counted as words.

\jformat{101010}

\begin{effectize}
\effect $PC \leftarrow PC + 4 * signext_{32}(offset)$
\end{effectize}

\subsection{JR}

The JR instruction performs an unconditional indirect jump to an absolute offset stored in a general-purpose register. It can be used for simple indirect jumps as well as to return from a subroutine.

\jrformat{101011}

\begin{effectize}
\effect $PC \leftarrow R_{dest}$
\end{effectize}

\subsection{JAL}

The JAL instruction stores the current \pc value in register \#31, then performs an unconditional direct jump to a relative immediate sign-extended 26-bit offset counted as words. It is primarily used for subroutine calls.

\jformat{101100}

\begin{effectize}
\effect $R_{31} \leftarrow PC$
\effect $PC \leftarrow PC + 4 * signext_{32}(offset)$
\end{effectize}

\subsection{JALR}

The JALR instruction remembers the current \pc value, then performs an unconditional indirect jump to an absolute offset stored in a general-purpose register. The previous PC value is then stored in register \#31. It is primarily used for indirect subroutine calls, such as virtual method invocations in object-oriented programming.

\jrformat{101101}

\begin{effectize}
\effect $returnAddress \leftarrow PC$
\effect $PC \leftarrow R_{dest}$
\effect $R_{31} \leftarrow returnAddress$
\end{effectize}



\section{Load and Store Instructions}

Load and store instructions transfer data from and to RAM and peripheral devices. All load/store instructions first compute a virtual address by adding a sign-extended 16-bit immediate value to a register value. That address is then transformed to a physical address by the \mmux. The load/store operation is sent to the SoC bus using the physical address and responded to by a slave device attached to the bus. Both the slave device itself and the target location inside that device are determined from the physical address. A write operation stores a value in a RAM location or device register, but may also trigger side-effects in some devices. Similarly, a read operation reads a value from a RAM location or device register, but may also trigger side-effects in some devices. Write operations take the data to write from a general-purpose register. Read operations store the received data in a general-purpose register.

All load/store operations must be aligned to the transferred data size. If a half-word (word) sized load/store operation is not half-word (word) aligned, it triggers an \name{Illegal Address Fault}.

All virtual addresses in the range 80000000$_h$ through FFFFFFFF$_h$ are privileged addresses and may only be accessed while in Kernel Mode. If such an address is accessed in User Mode, a \name{Privileged Address Fault} occurs.

The transformation of a virtual address to a physical address is done by the \mmu and may trigger a \name{TLB Miss Fault}, \name{TLB Invalid Fault} or \name{TLB Write Fault}. The service routine for these kinds of faults typically restarts the load/store operation after fixing the problem.

Any of these exceptions -- \name{Illegal Address Fault}, \name{Privileged Address Fault}, \name{TLB Miss Fault}, \name{TLB Invalid Fault} and \name{TLB Write Fault} -- causes the faulting address to be loaded into the \name{TLB Bad Address Register} ($S_4$).

Certain physical addresses may not actually correspond to any device attached to the SoC bus. This includes {\it holes} in the physical address map as well as the range of unused physical addresses (40000000$_h$ through FFFFFFFF$_h$). Access to such addresses results in a \name{Bus Timeout Fault}.

Load/store operations come in variants with different transfer size. Only the RAM and ROM support half-word and byte sized operations. Peripheral devices only support word-sized operations. Accessing peripheral devices with half-word or byte sized operations has an undefined effect. Access to RAM or ROM with different transfer sizes provides word-sized, half-word sized, and byte-sized views on the same memory locations. These views are arranged in a big-endian fashion.

When a half-word or byte sized location in RAM or ROM is read, the resulting value is extended to 32 bits to fit into a general-purpose register. Half-word and byte sized load operations come in variants that either sign-extend or zero-extend these values.

\newcommand{\badaddress}[1]{
	\begin{effectblock}
		\effect $S_4 \leftarrow A_v$
		\effect trigger a \name{#1}
	\end{effectblock}
	\effect end if
}
\newcommand{\signedload}{The result is sign-extended to 32 bits.}
\newcommand{\unsignedload}{The result is zero-extended to 32 bits.}
\newcommand{\signedloadeffect}{\effect $R_r \leftarrow signext_{32}($response value$)$}
\newcommand{\unsignedloadeffect}{\effect $R_r \leftarrow zeroext_{32}($response value$)$}
\newcommand{\ldstdesc}[6]{
\pagebreak
\subsection{#1}

The #1 instruction #3 a #4-sized value #5 RAM, ROM, or a peripheral device. #6

\ldstformat{#2}

}

\newcommand{\xaligned}[1]{
\effect if $A_v$ is not #1 aligned then \badaddress{Illegal Address Fault}
}
\newcommand{\haligned}{\xaligned{half-word}}
\newcommand{\waligned}{\xaligned{word}}
\newcommand{\writeprotection}{
\effect if the TLB entry for $pageNumber$ does not have the {\it write} bit set then \badaddress{TLB Write Fault}
}
\newcommand{\sendload}[1]{
\effect send a load #1 request using the address $A_p$ to the SoC bus
}
\newcommand{\sendstore}[2]{
\effect send a store #1 request using the address $A_p$ and data $#2$ to the SoC bus
}
\newcommand{\ldsteffects}[4]{
\begin{effectize}
\effect $A_v \leftarrow R_x + signext_{32}(y)$
#1
\effect if $A_{v,31} = 1$ and $U_C = 1$ then \badaddress{Privileged Address Fault}
\effect $pageNumber \leftarrow A_{v,31..12}$
\effect if no TLB entry exists for $pageNumber$ then \badaddress{TLB Miss Fault}
\effect if the TLB entry for $pageNumber$ does not have the {\it valid} bit set then \badaddress{TLB Invalid Fault}
#2
\effect $A_p \leftarrow $ page frame number in the TLB entry for $pageAddress$
#3
\effect if no response is received then trigger a \name{Bus Timeout Fault}
#4
\end{effectize}
\vfill

}

\ldstdesc{LDW}{110000}{reads}{word}{from}{}
\ldsteffects{\waligned}{}{\sendload{word}}{\effect $R_r \leftarrow$ response value}
\ldstdesc{LDH}{110001}{reads}{half-word}{from}{\signedload}
\ldsteffects{\haligned}{}{\sendload{half-word}}{\signedloadeffect}
\ldstdesc{LDHU}{110010}{reads}{half-word}{from}{\unsignedload}
\ldsteffects{\haligned}{}{\sendload{half-word}}{\unsignedloadeffect}
\ldstdesc{LDB}{110011}{reads}{byte}{from}{\signedload}
\ldsteffects{}{}{\sendload{byte}}{\signedloadeffect}
\ldstdesc{LDBU}{110100}{reads}{byte}{from}{\unsignedload}
\ldsteffects{}{}{\sendload{byte}}{\unsignedloadeffect}

\ldstdesc{STW}{110101}{writes}{word}{to}{}
\ldsteffects{\waligned}{\writeprotection}{\sendstore{word}{R_r}}{}
\ldstdesc{STH}{110110}{writes}{half-word}{to}{}
\ldsteffects{\haligned}{\writeprotection}{\sendstore{half-word}{R_{r,15..0}}}{}
\ldstdesc{STB}{110111}{writes}{byte}{to}{}
\ldsteffects{}{\writeprotection}{\sendstore{byte}{R_{r,7..0}}}{}



\section{System Instructions}

\subsection{TRAP}

The TRAP instruction triggers a \name{Trap Fault}. It is typically used by user programs to request action from the operating system.

System implementer's note: The fault mechanism causes general purpose register \#30 to be loaded with the address of the faulting instruction, that is, the TRAP instruction itself. However, the fault service routine typically wants to return to the instruction immediately following the TRAP instruction, such that the TRAP is not executed again. This can be achieved by adding 4 to the return address in $R_{30}$ in the service routine. See \myref{2}{rfx} for details.

\noargformat{101110}

\begin{effectize}
\effect trigger a \name{Trap Fault}
\end{effectize}

\subsection{RFX}

The RFX instruction returns control from an exception service routine to the interrupted program. The return address is taken from general purpose register \#30. The RFX instruction also restores the privilege mode and interrupt enable to the interrupted state by popping the topmost values of the corresponding state stacks in the \pswx. See \myref{2}{psw} and \myref{2}{rfx} for details.

\noargformat{101111}

\begin{effectize}
\priveffect
\effect $PC \leftarrow R_{30}$
\effect $I_C \leftarrow I_P$
\effect $I_P \leftarrow I_O$
\effect $U_C \leftarrow U_P$
\effect $U_P \leftarrow U_O$
\end{effectize}

\subsection{MVFS}

The MVFS transfers the value of a special-purpose register into a general-purpose register. See \myref{2}{special_purpose_registers} for details on the special-purpose registers.

\mvspformat{111000}

\begin{effectize}
\priveffect
\effect If $z$ does not denote a valid special-purpose register, then trigger an \name{Illegal Instruction Fault}
\effect $R_r \leftarrow S_z$
\end{effectize}

\subsection{MVTS}

The MVFS transfers the value of a general-purpose register into a special-purpose register. See \myref{2}{special_purpose_registers} for details on the special-purpose registers.

\mvspformat{111001}

\begin{effectize}
\priveffect
\effect If $z$ does not denote a valid special-purpose register, then trigger an \name{Illegal Instruction Fault}
\effect $S_z \leftarrow R_r$
\end{effectize}

\subsection{TBS}

The TBS instruction searches the TLB for a mapping for a virtual address specified in the TLB Entry High register ($S_2$) and stores the resulting entry index in the TLB Index register ($S_1$).

\noargformat{111010}

\begin{effectize}
\priveffect
\effect $PageNumber \leftarrow S_{2,31..12}$
\effect if the TLB contains an entry for $PageNumber$ then
\begin{itemize}
\item[] $S_1 \leftarrow$ (the corresponding TLB entry index)
\end{itemize}
\effect else
\begin{itemize}
\item[] $S_1 \leftarrow 80000000_{h}$
\end{itemize}
\end{effectize}

\textit{Special cases:} The TBS instruction will ``find'' a TLB entry that uses a page number in the direct-mapped virtual address space (C0000000$_h$ through FFFFFFFF$_h$) if the TLB Entry High register contains the corresponding page number. Normal address translation would not find such an entry since it always chooses direct mapping for such addresses.

\subsection{TBWR}

The TBWR instruction replaces a random TLB entry. First, the index of the entry to replace is determined as a random number in the range of non-fixed TLB entries (see \myref{2}{tlb_random}). Then, data from the TLB Entry High and Low registers ($S_2$ and $S_3$) is written into that TLB entry.

\noargformat{111011}

\begin{effectize}
\priveffect
\effect $X :=$ (random MOD 28) + 4
\effect TLB Entry \#$X \leftarrow (S_3, S_2)$
\end{effectize}

\subsection{TBRI}

The TBRI instruction reads data from a TLB entry indicated by the TLB Index register ($S_1$) into the TLB Entry High and Low registers ($S_2$ and $S_3$).

\noargformat{111100}

\begin{effectize}
\priveffect
\effect $X := S_1$ MOD 32
\effect $(S_3, S_2) \leftarrow$ TLB Entry \#$X$
\end{effectize}

\subsection{TBWI}

The TBWI instruction writes data from the TLB Entry High and Low registers ($S_2$ and $S_3$) into a TLB entry indicated by the TLB Index register ($S_1$).

\noargformat{111101}

\begin{effectize}
\priveffect
\effect $X := S_1$ MOD 32
\effect TLB Entry \#$X \leftarrow (S_3, S_2)$
\end{effectize}

